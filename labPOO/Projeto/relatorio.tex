\documentclass[11pt]{article}
\usepackage[utf8]{inputenc}

%Gummi|065|=)
\title{\textbf{Gerenciamento de um time de futebol: uma abordagem em Java}}
\author{Lucas Villela Canôas}
\date{31 de julho de 2019}
\begin{document}

\maketitle

\section{Objetivos}

A ideia consiste em criar um programa em java que faça a gestão de um time de futebol, se torna necessário realizar o registro de todos os funcionários que prestam serviço ao clube (massagistas, presidente do clube, treinador, jogadores, limpeza e outros), foi especificado para um mínimo de 5 rodadas de armazenamento de jogos anteriores na memória. Ao fim do registro, o programa deve também exibir o relatório financeiro do time.

Além disso, foi exigido no trabalho que se utilizasse todos os tipos de poliformismo expostos na aula, que foram sobrecarga(Overload), sobreposição (Override), de parâmetros e coerção.
Para refletir como um time funciona, observei a estrutura de do time Sport Club Corinthians Paulista.

\section{Ferramentas utilizadas}
Para a realização do trabalho foi utilizado o editor de texto Vim em um ambiente da distribuição Debian com um kernel Linux e o compilador javac.

\section{Problemas esperados}
Além dos problemas de utilização de estruturas relativamente desconhecidas pelo autor que não foram abordadas em sala devido a amplitude da linaguagem e o tempo finito, possivelmente os problemas de comparação e mitigação de inconsistências nas informações dos jogos serão um problema.

\section{Problemas encontrados}

O primeiro problema encontrado foi utilizar a estrutura enum, pois não havia sido criada o constrututor, o que gerou um erro inesperado, mas que rapidamente se resolveu.
Por algum motivo desconhecido, não foi possível aplicar o construtor e o nome da função principal da classe teve que ser alterada. Ainda assim, até o presente momento a compilação não pode ser realizada, pois existe uma variável não encontrada pelo compilador javac, o que depois se transformou em um erro na utilização do construtor que não foi encontrada a solução. Para contornar o problema, não utilizei construtor nas classes.
Não foi possível realizar a compilação do relatório, pois se tornou um procedimento confuso a extração e comparação usando a estrutura enum.



\end{document}
